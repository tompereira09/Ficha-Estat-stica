\documentclass[12pt]{article}

\title{Ficha Estatística}
\author{Tomás Pereira}

\begin{document}
\maketitle

1.\\\\
$\frac{7730+7200+7140+6960+3740+k}{6}=6225\equiv\frac{32770+k}{6}=\frac{37350}{6}\equiv32770+k=37550\\$$k=37550-32770\equiv k=4580$\\

2.\\\\
1) 2015\\
2) 2019\\
3) 2020\\

3.\\\\
$\frac{a+22200}{2}=22000\equiv\frac{a+22200}{2}=\frac{44000}{2}\equiv a+22200=44400\equiv\\ \equiv a=44000-22200\equiv a=21800$\\

4.\\\\
A: O vestuário de desporto é apresentado como maior que o calçado de\\desporto, quando na verdade é o contrário.\\
B: A soma das percentagens é maior que 100\%.\\\\\\\\\\\\

5.\\\\
$\frac{770+2900+1500+262+1000+k}{6}=1122\equiv\frac{6432+k}{6}=1122\equiv\frac{6432+k}{6}=\frac{6732}{6}\equiv\\\equiv6432+k=6732\equiv k=6732-6432\equiv k=300$\\

6.\\\\
1) AML\\
2) Alentejo\\
3) Centro\\

7.\\\\
$\frac{20+18+14+16+11+16+8+9}{8}=14$\\\\
R: B.\\

8.\\\\
a) Finlândia\\
b) Estónia\\
c) Grécia\\

9.\\\\
1)2017\\
2)2014\\
3)2019\\

10.\\\\
c = \{27;34;34;40;47;48;51;57;58\}\\
$\frac{n+1}{2}$\\
p = $\frac{9+1}{2}=\frac{10}{2}=5$\\
$\tilde{x}$ = c[p] = $47$\\\\
R: C.\\

11.\\\\
1)2013\\
2)2020\\
3)2017\\

12.\\\\
$\overline{x}=\frac{12+13+17+18+22+20+21+21}{8}=\frac{144}{8}=18$\\\\
R: $18\;m^3$.\\



\end{document}
